\documentclass[french]{article}

\usepackage[utf8]{inputenc}
\usepackage[top=2cm,bottom=1.75cm, left=1.5cm, right=1.75cm,a4paper]{geometry}
\usepackage{empheq}
\usepackage{fancybox,xcolor}

\usepackage{pgfplots}
\usepackage{pgfplotstable}
\usepackage{graphicx}
\pgfplotsset{compat=1.5}

\usepackage[squaren, Gray, cdot]{SIunits}

\renewcommand{\thesection}{\Roman{section}.}
\renewcommand{\thesubsection}{\Alph{subsection}.}
\renewcommand{\thesubsubsection}{\arabic{subsubsection}.}
\renewcommand{\theparagraph}{\alph{paragraph}.}

\title{Rapport - TIPE - Lifters}
\author{Mathieu BOUR}
\date{Mai 2017}

\begin{document}
	\maketitle
	
	L'objectif initial de cette étude était de mettre en lumière l'influence des paramètres de construction des lifters sur la masse transportable, c'est-à-dire la charge maximale pouvant être ajoutée au lifter tel qu'il décolle.
	
	J'ai réalisé 3 expériences pour déterminer l'importance des paramètres de construction des lifters sur leurs efficacité, respectivement :
	\begin{itemize}
		\item Variation de l'épaisseur de la bande $e$
		\item Variation du rayon du fil de cuivre $r$
		\item Variation du nombre de côtés $n$
	\end{itemize}
	
	\subsection{Variation de l'épaisseur de la bande $e$}
	10 lifters triangulaires d'épaisseur $e$ allant de 3 $\centi\meter$ à 7.5 $\centi\meter$, par pas de 0.5 $\centi\meter$ ont été construits, les résultats sont si affichés sur la figure 1. Au-delà de 7.5 $\centi\meter$, le lifter ne décolle plus. Les différence de masse liées aux différences de matière dûes aux variations des épaisseurs des bandes ont été compensées.
	
	\subsection{Variation du rayon du fil $r$}
	4 lifters on été réalisés, à $e = $ 4.5 $\centi\meter$ constant, les rayons du fil de cuivre utilisés variant de 0.025 $\milli\meter$ à 0.1 $\milli\meter$, les résultats sont affichés en figure 2. Au-delà de $r = 0.1 \milli\meter$, le lifter ne décolle plus. Les différences de masse entre ces lifters n'ont pas été compensées, car elles sont négligeables.
	
	\pagebreak
	\begin{tikzpicture}
		\begin{axis}
			[
				width = \textwidth,
				height = \axisdefaultheight,
				title = Charge utilie $m_c$ en fonction de l'épaisseur $e$,
				xlabel = Épaisseur de la bande ($\centi\meter$),
				ylabel = Charge utile ($\gram$),
				ymin = 0,
				legend pos = outer north east,
			]
			\addplot[
				only marks,
				mark = *
			]
			table [
				x = e,
				y = mc,
				col sep = comma,
			]
			{exps/epaisseur.csv};
		\end{axis}
	\end{tikzpicture}

	\begin{tikzpicture}
		\begin{axis}
			[
				width = \textwidth,
				height = \axisdefaultheight,
				title = Charge utile $m_c$ en fonction du rayon $r$,
				xlabel = Rayon du fil ($\milli\meter$),
				ylabel = Charge utile ($\gram$),
				ymin = 0,
				legend pos = outer north east,
				tick label style={/pgf/number format/fixed}
			]
			\addplot[
				only marks,
				mark = *
			]
			table [
				x = r,
				y = mc,
				col sep = comma,
			]
			{exps/rayon.csv};
		\end{axis}
	\end{tikzpicture}
\end{document}